\documentclass[a4paper, twoside]{article}

\usepackage[ngerman]{babel}
\usepackage{mathtools}
\usepackage[margin=2.5cm]{geometry}


\title{Formelsammlung---Numerische Methoden}
\author{Tim Hilt \and Emil Slomka}
\date{\today}

\begin{document}

\maketitle
\tableofcontents

\section{Lineare Gleichungssysteme}

Die unten beschriebenen Verfahren suchen Lösungen für die $x$-Werte. 

\subsection{Jacobi-Iteration}

\subsubsection{Jacobi-Iteration in Matrix-Vektor-Notation}

$\mathbf{L}$: Lower part of Matrix

$\mathbf{D}$: Extracted Diagonal

$\mathbf{U}$: Upper part of Matrix

\[\mathbf{x}^{(k+1)} = -\mathbf{D}^{-1} (\mathbf{L} + \mathbf{U})\mathbf{x}^{(k)} + \mathbf{D}^{-1}\mathbf{b}\]

\textbf{Achtung}: Wenn bei der Jacobi-Iteration alle Startwerte $=0$ sind muss \textbf{Nur der zweite Term} $\mathbf{D}^{-1}\mathbf{b}$ betrachtet werden!!!

\subsubsection{Vorgehen}

\begin{enumerate}
\item Stelle einzelne Gleichungen auf
\item Auflösen nach den Variablen der jeweiligen Zeile
\item Links steht jetzt die Variable der nächsten Iteration, rechts stehen die vorhergehenden Werte.
\item Gleichungen ausrechnen
\end{enumerate}

\subsection{Konvergenz}

Die untenstehenden Kriterien stellen das Konvergenzkriterium für beide Iterationsverfahren dar. Es gilt sowohl für das Jacobi- als auch für das Gauss-Seidel-Verfahren und für beliebige Startwerte.

\subsubsection{Diagonaldominanz}

Eine Matrix ist dann diagonaldominant, wenn in allen Zeilen der Betrag des Diagonalelements der Matrix größer ist als die Summe des Betrages der restlichen Elemente.

\[\sum_{j \neq k} |a_{ij}| < |a_{kk}| \ \text{für}\ k = 1, \ldots , n.\]

\subsubsection{Spektralradius}

\[\rho(\mathbf{A}) = \max_{j=1,\ldots,n} |\lambda_{j}| = \max(|-\mathbf{D}^{-1}(\mathbf{L} + \mathbf{U})|)\]

\(\Rightarrow\) Die Iteration konvergiert, wenn \(|\rho(\mathbf{A})| < 1\)

\(\Rightarrow\) Je kleiner der Spektralradius, desto schneller die Konvergenz

\subsection{Gauss-Seidel-Iteration}

\section{Nicht-lineare Gleichungssysteme}

\section{Interpolation und Approximation}

\section{Numerische Integration}

\section{Optimierung}

\section{Gewöhnliche Differenzialgleichungen}

\end{document}